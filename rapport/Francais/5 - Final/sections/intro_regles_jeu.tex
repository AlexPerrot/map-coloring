\section{Map-coloring game}

Le sujet de l'article est une variante de la coloration de graphe impliquant deux joueurs. La règle de coloration est la même, chaque sommet ne peut avoir la même couleur qu'un de ses voisins. 

Le premier joueur essaie de colorier correctement le graphe. Le second essaie de l'en empêcher sans pour autant pouvoir outrepasser la règle de coloration. 
Au début du jeu, on définit un nombre N de couleur qui pourrons être utilisé lors de la partie. La partie s'arrête si tous les sommets sont coloriés ou alors si il n'est plus possible de colorier le graphe sans ajouter une N+1 eme couleur, ce qui serait contraire aux régles.

 
On définit le "game chromatic number" comme le nombre minimal de couleurs avec lesquelles le premier joueur peut réussir à achever une coloration correcte du graphe quelque soit la stratégie du second.
