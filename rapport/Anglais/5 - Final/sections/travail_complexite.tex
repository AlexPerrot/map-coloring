\section{Complexity}

In this section, we present a short (and probably inaccurate) study of the space and time complexity of our implementation.

\subsection{Space complexity of the search tree}

We will first try to calculate how many different games are possible on a given graph. We will call $v$ its number of vertices and $c$ the number of colors used for the game. We have $c\leq v$.

Lets think about the way a map-coloring is played and the number of choices for each move. On the first move, $v$ vertices can be colored with $c$ colors, leading to $vc$ choices. On the second move, $v-1$ vertices can be colored using $c$ colors, for $c(v-1)$ choices. Up until the last move, where $(v-(v-1))=1$ can be colored with $c$ colors. The total number of possible different games is :
\[
cv*c(v-1)*c(v-2)*...*c2*c1 =
\prod_{i=0}^{i<v}c(v-i) =
c^v \prod_{i=0}^{i<v}v-i = c^v*v!
\]

Considering the fact that introducing a new color in the game is isomorphic regardless of the color chosen, we can reduce the number of colors for a move to the number of colors already used plus one (the new color), up until $c$. This leads to the following calculations :
\[
1v*2(v-1)*...*c2*c1 = \prod_{i=0}^{i<c} (i+1)(v-i) \prod_{i=c}^{i<v} c(v-i)
\]\[
= c!c^{v-c} \prod_{i=0}^{i<v}(v-i) = c!c^{v-c}v!
\]