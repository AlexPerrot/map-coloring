\section{Map-coloring game}

The studied article's subject defines a variant of the graph coloring involving two players.

The first player tries to color the graph correctly. The second tries to hinder and stop him without overriding the coloring rule. The coloring rule stays the same: each vertex cannot have the same color as its neighbors. 
At the beginning of the game, we set a number N of colors than can be used in the game. The game ends if all the vertices are colored or if it is not possible to color the graph without adding a color while exceeding N, which is contrary to the rules.

 
We defines the \textit{game chromatic number} as the minimal number of colors with which the first player can achieve a correct coloring of the graph whatever the strategy of the second player.
