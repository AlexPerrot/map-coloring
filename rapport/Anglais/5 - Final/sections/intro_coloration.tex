\section{Coloration de Graph}

La coloration de graphe est un problème qui date du XIXème siècle.
Il s'agit d'assigner à chaque sommet d'un graphe des couleurs afin que deux sommets adjacents (reliés par une arête) ne soient pas de la même couleur. Il existe des variantes consistant à colorier les arêtes ou les faces d'un graphe, mais elle se ramènent toutes à une coloration de sommets. Le nombre minimal de couleurs pour obtenir une bonne coloration sur un graphe donné est appelé le nombre chromatique de ce graphe. Un graphe est dit k-coloriable si son nombre chromatique est k. Décider si un graphe est k-coloriable est un problème NP-complet.
