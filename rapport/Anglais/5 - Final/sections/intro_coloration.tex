\section{Graph coloring}

Graph coloring is not a new problem in mathematics and computer science. The first results in this field date back to 1852, when \emph{Francis Guthrie} postulated the famous four color theorem\footnote{He did not write any publication on this theorem.}.

Coloring a graph consists of assigning a color to each vertex of the graph, such that two adjacent vertices are of different colors. A such coloring is called a proper coloring. There are other forms of coloring, such as edge or face coloring, but they always reduce to a vertex coloring on another graph.

A graph is said to be k-coloriable if it admits a proper coloring with k colors. The minimal number of colors needed to achieve a proper coloring of a particular graph is called the \textit{chromatic number} of that graph, noted $ \chi (g)$. Deciding if a graph is k-coloriable for a given value of k is NP-complete.
