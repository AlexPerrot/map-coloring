\section{Graph coloring}

Graph coloring is not a new problem in mathematics and computer science. The first results in the field date back to 1852, when Francis Guthrie postulated the famous four color theorem.

Coloring a graph consists of assigning a color to each vertex of the graph, such that two adjacent vertices are of different colors. A such coloring is called a proper coloring. There are other forms of coloring, such as edge or face coloring, but they always reduce to a vertex coloring on another graph.

 A graph is said to be k-coloriable if it admits a proper coloring with k colors. The minimal number of colors needed to achieve a proper coloring of a particular graph is called the chromatic number of that graph, noted $ \chi (g)$. Deciding if a graph is k-coloriable for a given value of k is NP-complete.


La coloration de graphe est un problème qui date du XIXème siècle.
Il s'agit d'assigner à chaque sommet d'un graphe des couleurs afin que deux sommets adjacents (reliés par une arête) ne soient pas de la même couleur. Il existe des variantes consistant à colorier les arêtes ou les faces d'un graphe, mais elle se ramènent toutes à une coloration de sommets. Le nombre minimal de couleurs pour obtenir une bonne coloration sur un graphe donné est appelé le nombre chromatique de ce graphe. Un graphe est dit k-coloriable si son nombre chromatique est <=k. Décider si un graphe est k-coloriable est un problème NP-complet.
