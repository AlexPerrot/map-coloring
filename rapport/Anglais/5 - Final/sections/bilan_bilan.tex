\section{Assessment}

Overall, we can say that the results are quite positive. Indeed, we were able to meet the original objectives and our results are better than what we expected.

We believe it is possible that there is a theoritical model which outperforms those presented in the article we studied.

\subsection{Experimental biases}

We have reason to believe that our protocols of experimentation can be improved to provide more reliable results:

\begin{itemize}
\item We can not make Alice or Bob stronger. If we increase the number of simulation performed by Alice, Bob will also increase the number of simulations that he performs, thus increasing its strength as much as that of Alice. Furthermore, the goal is to see whether Alice can win regardless of the strength of Bob, but our Bob is not infaillible.
\item Unlike the strategies of the article, our AI adapts itself based on the simulations it performs. Where a theoretical model always redo the same thing which seems to be the best choice, our AI adapts itself and will \textit{know} if a move leads to a victory or not, thus being able to out of an impasse.
\end{itemize}


